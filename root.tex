%%%%%%%%%%%%%%%%%%%%%%%%%%%%%%%%%%%%%%%%%%%%%%%%%%%%%%%%%%%%%%%%%%%%%%%%%%%%%%%%
%2345678901234567890123456789012345678901234567890123456789012345678901234567890
%        1         2         3         4         5         6         7         8

\documentclass[letterpaper, 10 pt, conference]{ieeeconf}  % Comment this line out if you need a4paper

%\documentclass[a4paper, 10pt, conference]{ieeeconf}      % Use this line for a4 paper

\IEEEoverridecommandlockouts                              % This command is only needed if 
                                                          % you want to use the \thanks command

\overrideIEEEmargins                                      % Needed to meet printer requirements.

% See the \addtolength command later in the file to balance the column lengths
% on the last page of the document

% The following packages can be found on http:\\www.ctan.org
\usepackage{graphics} % for pdf, bitmapped graphics files
\usepackage{epsfig} % for postscript graphics files
\usepackage{mathptmx} % assumes new font selection scheme installed
\usepackage{times} % assumes new font selection scheme installed
\usepackage{amsmath} % assumes amsmath package installed
\usepackage{amssymb}  % assumes amsmath package installed
\usepackage{amsthm}

\newtheorem{myTheo}{Theorem}

\theoremstyle{remark}
\newtheorem{myrem}{Remark}

\title{\LARGE \bf
Geometric Part*
}


\author{Xiao Ming Xiu$^{1}$  \\% <-this % stops a space
\\
\footnotesize\textit{\today}
\thanks{*This work was supported by Standard-Robots Ltd.,co.}% <-this % stops a space
\thanks{$^{1}$Xiao Ming Xiu is with Visual SLAM Algorithm Group, Standard-Robots Ltd.,co. Shenzhen, China. 
        {\tt\small stevenxiu@aliyun.com}}%
}


\begin{document}



\maketitle
\thispagestyle{empty}
\pagestyle{empty}


%%%%%%%%%%%%%%%%%%%%%%%%%%%%%%%%%%%%%%%%%%%%%%%%%%%%%%%%%%%%%%%%%%%%%%%%%%%%%%%%
\begin{abstract}
Given 3 2-D points' coordinates in two different Reference Systems, Barcode System (BS) and Camera System (CS), output two things:  $\vec{t}$ and $\theta$, defined as the CS origin's coordinates in BS and the rotated angle of CS relative to BS.  

\end{abstract}


%%%%%%%%%%%%%%%%%%%%%%%%%%%%%%%%%%%%%%%%%%%%%%%%%%%%%%%%%%%%%%%%%%%%%%%%%%%%%%%%
\section{Problem Setting}
\subsection{Given}
Notation: $$ BS --- S $$  $$ CS --- S' $$

Three Points' coordinates in both systems are known:
$$ P1:  (x_1,y_1), (x_1',y_1') $$
$$ P2:  (x_2,y_2), (x_2',y_2') $$
$$ P3:  (x_3,y_3), (x_3',y_3') $$

\subsection{Want}
$$\vec{t} = (t_x, t_y)$$
$$ \theta $$

where, \\$\vec{t}$ is defined by S' origin in S.\\
$\theta$ is defined by

\begin{equation}
\begin{aligned}
(\epsilon_x',\epsilon_y')&= (\epsilon_x,\epsilon_y)*R\\
						&= (\epsilon_x,\epsilon_y)*
\begin{bmatrix}
\cos{\theta}&-\sin{\theta}\\
\sin{\theta}&cos{\theta}
\end{bmatrix}
\end{aligned}
\end{equation}

($\epsilon_x, \epsilon_y$ denotes the basis vector in S.  $\epsilon_x', \epsilon_y'$ denotes the basis vector in S'.)

\subsection{Relationship Equation}

Denote $\vec{P}$ (column vector)as vector coordinate representation in S and $\vec{P}'$ in S'. We have

\begin{equation}
\vec{P} = \vec{t} + R*\vec{P}' 
\end{equation}

Write in explicit form:

\begin{equation}
\begin{bmatrix}
x\\
y
\end{bmatrix}
=
\begin{bmatrix}
t_x \\
t_y
\end{bmatrix}
+
\begin{bmatrix}
\cos{\theta}&-\sin{\theta}\\
\sin{\theta}&cos{\theta}
\end{bmatrix}
*
\begin{bmatrix}
x' \\
y'
\end{bmatrix}
\end{equation}

Eq. (3) is necessary and sufficient.\\ 
You have 3 unknown variables, $t_x, t_y, \theta$. Given 1 pair coordinates $\vec{P}$ and $\vec{P}'$, you will get 2 independent equations. At least,  2 points are needed to solve all the 3 variables, though 1 equation is over determined. \\

\section{Question: what is minimal set of input sufficient to solve?}
\begin{myTheo}
Given 6 Rits of $x_1,y_1,x_1',y_1',x_2,x_2'$ is sufficient.
\end{myTheo}
3 independent equations have already been prepared up from Eq.(3). Only left with a Two-fold ambiguity.\\

\begin{myrem}
The set of 1.5 points, not 2,  is the minimum input.  
\end{myrem}

\begin{myrem}
Do there exist the following case: the input given format cannot exactly fit the sufficient and necessary solution of Problem, at least have some input as overdetermined condition and called unavoidable waste input?
\end{myrem}


\section{Solution}
Eliminate $\theta$ first from Eq. (3) by Left Multiply (based on $R^T*R=I$) :
( Eq. (3) represents 2 equations.)
\begin{equation}
(x-t_x)^2 + (y-t_y)^2 = {x'}^2 + {y'}^2
\end{equation}

Rely on Point P1 and P2.  Substitute Eq.(4) as:
\begin{equation}
(x_1-t_x)^2 + (y_1-t_y)^2 = {x_1'}^2 + {y_1'}^2 
\end{equation}
\begin{equation}
(x_2-t_x)^2 + (y_2-t_y)^2 = {x_2'}^2 + {y_2'}^2
\end{equation}

Use Eq.(5)- Eq.(6), we get

\begin{equation}
t_y=k t_x + b
\end{equation}

where
\begin{equation}
k=-\frac{x_1 - x_2}{y_1 - y_2}
\end{equation}
\begin{equation}
\begin{aligned}
b=&{\frac{1}{2}}*\{ y_1 + y_2 \\
&- {\frac{1}{y_1 - y_2}}[x'_1^2 + y'_1^2 - x'_2^2 -y'_2^2 -x_1^2 + x_2^2]\}
\end{aligned}
\end{equation}

Substitute $t_y$ in Eq. (5)
\begin{equation}
At_x^2 + Bt_x + C =0
\end{equation}
Where,
\begin{equation}
\begin{aligned}
A&=1+k^2 \\
B&=-2[x_1 + k(y_1 -b)]\\
C&= x_1^2 + (y_1 -b)^2 - x_1'^2 - y_1'^2
\end{aligned}
\end{equation}

Solve this quadratic equation
$$\Delta = B^2 - 4AC$$

\begin{equation}
t_x= \frac{-B \pm \sqrt{\Delta} }{2A}
\end{equation}

The corresponding $t_y$ can be derived from Eq. (7).

Here is a two-fold ambiguity. Only one of the roots are correct. Check them by the third point P3:

\begin{equation}
(x_3-t_x)^2 + (y_3 - t_y)^2 = x'_3^2 + y'_3^2
\end{equation}

Define
\begin{equation}
\delta_{check} = |(x_3-t_x)^2 + (y_3 - t_y)^2  - ( x'_3^2 + y'_3^2 )| ?=? 0
\end{equation}


The right root will satisfy Eq. (13). In  practice we can set the threshold of $\delta_{check}$.

\section{Experiments of Oct.17}
In the evening of Oct.17, a 2cm translation experiment was conducted.

    \begin{figure}  
    \begin{minipage}[t]{0.5\linewidth}  
    \centering  
    \includegraphics[width=1.6in]{0cm1017.jpg}  
    \caption{0cm Case No Lean}  
    \label{fig:side:a}  
    \end{minipage}%  
    \begin{minipage}[t]{0.5\linewidth}  
    \centering  
    \includegraphics[width=1.6in]{2cm1017.jpg}  
    \caption{2cm Case No Lean}  
    \label{fig:side:b}  
    \end{minipage}  
    \end{figure}  

\subsection{2cm case data}

Notice you need to transform the S' pixel unit to cm.
 
\begin{table}[h]
\caption{2cm1017.jpg's P1,P2,P3 }
\label{table_example}
\begin{center}
\begin{tabular}{|c|c|c|}

\hline
2cmNoLean & BS (cm)& CS(pixel)\\
\hline

P1 & (0,0)& (267,455) \\
\hline

P2 & $(3*cos(30^\circ),3*sin(30^\circ))$ & (271,256) \\

\hline
P3 & $(-3*cos(30^\circ),3*sin(30^\circ))$ & (466,460)\\
\hline
\end{tabular}
\end{center}
\end{table}

Output:

$$t_x=7.94855$$
$$t_y=-0.215983$$
$$\theta=120^\circ$$


\subsection{0cm case data}

 
\begin{table}[h]
\caption{0cm1017.jpg's P1,P2,P3 }
\label{table_example}
\begin{center}
\begin{tabular}{|c|c|c|}

\hline
0cmNoLean & BS (cm)& CS(pixel)\\
\hline

P1 & (0,0)& (399,458) \\
\hline

P2 & $(3*cos(30^\circ),3*sin(30^\circ))$ & (406,257) \\

\hline
P3 & $(-3*cos(30^\circ),3*sin(30^\circ))$ & (599,463)\\
\hline
\end{tabular}
\end{center}
\end{table}

Output:

$$t_x=8.82632$$
$$t_y=-2.04685$$
$$\theta=118.005^\circ$$

\subsection{Translation Calculation}
\begin{equation}
\begin{aligned}
Translation&=\sqrt{ (7.94855-8.82632)^2 +(-0.215983 + 2.04685)^2}\\
		   &=2.03 cm
\end{aligned}
\end{equation}

The real translation is just 2 cm. So the precision has attained to 0.3mm. 


\section{Experiments of Oct.19}
In the evening of Oct.19, just use the data collected in Sep., to check the algorithm again in Lean case.

    \begin{figure}  
    \begin{minipage}[t]{0.5\linewidth}  
    \centering  
    \includegraphics[width=1.6in]{0cm96Lean.jpg}  
    \caption{0cm Case No Lean}  
    \label{fig:side:a}  
    \end{minipage}%  
    \begin{minipage}[t]{0.5\linewidth}  
    \centering  
    \includegraphics[width=1.6in]{1cm96Lean.jpg}  
    \caption{2cm Case No Lean}  
    \label{fig:side:b}  
    \end{minipage}  
    \end{figure}  





\end{document}
